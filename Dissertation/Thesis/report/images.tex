\batchmode


\usepackage[dvips]{color}


\pagecolor[gray]{.7}

\usepackage[latin1]{inputenc}



\makeatletter

\makeatletter
\count@=\the\catcode`\_ \catcode`\_=8 
\newenvironment{tex2html_wrap}{}{}%
\catcode`\<=12\catcode`\_=\count@
\newcommand{\providedcommand}[1]{\expandafter\providecommand\csname #1\endcsname}%
\newcommand{\renewedcommand}[1]{\expandafter\providecommand\csname #1\endcsname{}%
  \expandafter\renewcommand\csname #1\endcsname}%
\newcommand{\newedenvironment}[1]{\newenvironment{#1}{}{}\renewenvironment{#1}}%
\let\newedcommand\renewedcommand
\let\renewedenvironment\newedenvironment
\makeatother
\let\mathon=$
\let\mathoff=$
\ifx\AtBeginDocument\undefined \newcommand{\AtBeginDocument}[1]{}\fi
\newbox\sizebox
\setlength{\hoffset}{0pt}\setlength{\voffset}{0pt}
\addtolength{\textheight}{\footskip}\setlength{\footskip}{0pt}
\addtolength{\textheight}{\topmargin}\setlength{\topmargin}{0pt}
\addtolength{\textheight}{\headheight}\setlength{\headheight}{0pt}
\addtolength{\textheight}{\headsep}\setlength{\headsep}{0pt}
\setlength{\textwidth}{349pt}
\newwrite\lthtmlwrite
\makeatletter
\let\realnormalsize=\normalsize
\global\topskip=2sp
\def\preveqno{}\let\real@float=\@float \let\realend@float=\end@float
\def\@float{\let\@savefreelist\@freelist\real@float}
\def\liih@math{\ifmmode$\else\bad@math\fi}
\def\end@float{\realend@float\global\let\@freelist\@savefreelist}
\let\real@dbflt=\@dbflt \let\end@dblfloat=\end@float
\let\@largefloatcheck=\relax
\let\if@boxedmulticols=\iftrue
\def\@dbflt{\let\@savefreelist\@freelist\real@dbflt}
\def\adjustnormalsize{\def\normalsize{\mathsurround=0pt \realnormalsize
 \parindent=0pt\abovedisplayskip=0pt\belowdisplayskip=0pt}%
 \def\phantompar{\csname par\endcsname}\normalsize}%
\def\lthtmltypeout#1{{\let\protect\string \immediate\write\lthtmlwrite{#1}}}%
\newcommand\lthtmlhboxmathA{\adjustnormalsize\setbox\sizebox=\hbox\bgroup\kern.05em }%
\newcommand\lthtmlhboxmathB{\adjustnormalsize\setbox\sizebox=\hbox to\hsize\bgroup\hfill }%
\newcommand\lthtmlvboxmathA{\adjustnormalsize\setbox\sizebox=\vbox\bgroup %
 \let\ifinner=\iffalse \let\)\liih@math }%
\newcommand\lthtmlboxmathZ{\@next\next\@currlist{}{\def\next{\voidb@x}}%
 \expandafter\box\next\egroup}%
\newcommand\lthtmlmathtype[1]{\gdef\lthtmlmathenv{#1}}%
\newcommand\lthtmllogmath{\dimen0\ht\sizebox \advance\dimen0\dp\sizebox
  \ifdim\dimen0>.95\vsize
   \lthtmltypeout{%
*** image for \lthtmlmathenv\space is too tall at \the\dimen0, reducing to .95 vsize ***}%
   \ht\sizebox.95\vsize \dp\sizebox\z@ \fi
  \lthtmltypeout{l2hSize %
:\lthtmlmathenv:\the\ht\sizebox::\the\dp\sizebox::\the\wd\sizebox.\preveqno}}%
\newcommand\lthtmlfigureA[1]{\let\@savefreelist\@freelist
       \lthtmlmathtype{#1}\lthtmlvboxmathA}%
\newcommand\lthtmlpictureA{\bgroup\catcode`\_=8 \lthtmlpictureB}%
\newcommand\lthtmlpictureB[1]{\lthtmlmathtype{#1}\egroup
       \let\@savefreelist\@freelist \lthtmlhboxmathB}%
\newcommand\lthtmlpictureZ[1]{\hfill\lthtmlfigureZ}%
\newcommand\lthtmlfigureZ{\lthtmlboxmathZ\lthtmllogmath\copy\sizebox
       \global\let\@freelist\@savefreelist}%
\newcommand\lthtmldisplayA{\bgroup\catcode`\_=8 \lthtmldisplayAi}%
\newcommand\lthtmldisplayAi[1]{\lthtmlmathtype{#1}\egroup\lthtmlvboxmathA}%
\newcommand\lthtmldisplayB[1]{\edef\preveqno{(\theequation)}%
  \lthtmldisplayA{#1}\let\@eqnnum\relax}%
\newcommand\lthtmldisplayZ{\lthtmlboxmathZ\lthtmllogmath\lthtmlsetmath}%
\newcommand\lthtmlinlinemathA{\bgroup\catcode`\_=8 \lthtmlinlinemathB}
\newcommand\lthtmlinlinemathB[1]{\lthtmlmathtype{#1}\egroup\lthtmlhboxmathA
  \vrule height1.5ex width0pt }%
\newcommand\lthtmlinlineA{\bgroup\catcode`\_=8 \lthtmlinlineB}%
\newcommand\lthtmlinlineB[1]{\lthtmlmathtype{#1}\egroup\lthtmlhboxmathA}%
\newcommand\lthtmlinlineZ{\egroup\expandafter\ifdim\dp\sizebox>0pt %
  \expandafter\centerinlinemath\fi\lthtmllogmath\lthtmlsetinline}
\newcommand\lthtmlinlinemathZ{\egroup\expandafter\ifdim\dp\sizebox>0pt %
  \expandafter\centerinlinemath\fi\lthtmllogmath\lthtmlsetmath}
\newcommand\lthtmlindisplaymathZ{\egroup %
  \centerinlinemath\lthtmllogmath\lthtmlsetmath}
\def\lthtmlsetinline{\hbox{\vrule width.1em \vtop{\vbox{%
  \kern.1em\copy\sizebox}\ifdim\dp\sizebox>0pt\kern.1em\else\kern.3pt\fi
  \ifdim\hsize>\wd\sizebox \hrule depth1pt\fi}}}
\def\lthtmlsetmath{\hbox{\vrule width.1em\kern-.05em\vtop{\vbox{%
  \kern.1em\kern0.8 pt\hbox{\hglue.17em\copy\sizebox\hglue0.8 pt}}\kern.3pt%
  \ifdim\dp\sizebox>0pt\kern.1em\fi \kern0.8 pt%
  \ifdim\hsize>\wd\sizebox \hrule depth1pt\fi}}}
\def\centerinlinemath{%
  \dimen1=\ifdim\ht\sizebox<\dp\sizebox \dp\sizebox\else\ht\sizebox\fi
  \advance\dimen1by.5pt \vrule width0pt height\dimen1 depth\dimen1 
 \dp\sizebox=\dimen1\ht\sizebox=\dimen1\relax}

\def\lthtmlcheckvsize{\ifdim\ht\sizebox<\vsize 
  \ifdim\wd\sizebox<\hsize\expandafter\hfill\fi \expandafter\vfill
  \else\expandafter\vss\fi}%
\providecommand{\selectlanguage}[1]{}%
\makeatletter \tracingstats = 1 
\providecommand{\Beta}{\textrm{B}}
\providecommand{\Mu}{\textrm{M}}
\providecommand{\Kappa}{\textrm{K}}
\providecommand{\Rho}{\textrm{R}}
\providecommand{\Epsilon}{\textrm{E}}
\providecommand{\Chi}{\textrm{X}}
\providecommand{\Iota}{\textrm{J}}
\providecommand{\omicron}{\textrm{o}}
\providecommand{\Zeta}{\textrm{Z}}
\providecommand{\Eta}{\textrm{H}}
\providecommand{\Nu}{\textrm{N}}
\providecommand{\Omicron}{\textrm{O}}
\providecommand{\Tau}{\textrm{T}}
\providecommand{\Alpha}{\textrm{A}}


\begin{document}
\pagestyle{empty}\thispagestyle{empty}\lthtmltypeout{}%
\lthtmltypeout{latex2htmlLength hsize=\the\hsize}\lthtmltypeout{}%
\lthtmltypeout{latex2htmlLength vsize=\the\vsize}\lthtmltypeout{}%
\lthtmltypeout{latex2htmlLength hoffset=\the\hoffset}\lthtmltypeout{}%
\lthtmltypeout{latex2htmlLength voffset=\the\voffset}\lthtmltypeout{}%
\lthtmltypeout{latex2htmlLength topmargin=\the\topmargin}\lthtmltypeout{}%
\lthtmltypeout{latex2htmlLength topskip=\the\topskip}\lthtmltypeout{}%
\lthtmltypeout{latex2htmlLength headheight=\the\headheight}\lthtmltypeout{}%
\lthtmltypeout{latex2htmlLength headsep=\the\headsep}\lthtmltypeout{}%
\lthtmltypeout{latex2htmlLength parskip=\the\parskip}\lthtmltypeout{}%
\lthtmltypeout{latex2htmlLength oddsidemargin=\the\oddsidemargin}\lthtmltypeout{}%
\makeatletter
\if@twoside\lthtmltypeout{latex2htmlLength evensidemargin=\the\evensidemargin}%
\else\lthtmltypeout{latex2htmlLength evensidemargin=\the\oddsidemargin}\fi%
\lthtmltypeout{}%
\makeatother
\setcounter{page}{1}
\onecolumn

% !!! IMAGES START HERE !!!



\setlength{\parindent}{0cm}%

\setlength{\parindent}{0cm}


\setlength{\parskip}{2ex}%

\setlength{\parskip}{2ex}

\renewcommand{\}{}
\stepcounter{chapter}
\stepcounter{section}
{\newpage\clearpage
\lthtmlfigureA{enumeratestar205}%
\begin{enumerate*}
 \item Become familiar with DSM techniques for Digital Subscriber Lines.
 \item Become familiar with the CUDA environment for GPU's and identify a suitable platform.
 \item Investigate efficient implementations of level 2 DSM.
 \item Develop an implementation of a level 2 bit-loading algorithm using GPU's.
\end{enumerate*}%
\lthtmlfigureZ
\lthtmlcheckvsize\clearpage}

{\newpage\clearpage
\lthtmlfigureA{enumeratestar208}%
\begin{enumerate*}
 \item Analyse the performance of your implementation in terms of speed, cost, and scalability (number of lines).
 \item Compare your design with existing implementations.
\end{enumerate*}%
\lthtmlfigureZ
\lthtmlcheckvsize\clearpage}

{\newpage\clearpage
\lthtmlfigureA{enumeratestar211}%
\begin{enumerate*}
 \item Understand how to use CUDA to programme GPU's.
 \item Be able to design bit-loading algorithms for DMT.
 \item How to analyse the performance of an implementation.
\end{enumerate*}%
\lthtmlfigureZ
\lthtmlcheckvsize\clearpage}

\stepcounter{section}
{\newpage\clearpage
\lthtmlfigureA{itemizestar215}%
\begin{itemize*}
 
\setlength{\itemsep}{0.75pt} 
 
\setlength{\parskip}{0pt} 
 
\setlength{\parsep}{0pt} 
\par
\item Problem Research
  \begin{itemize*}   \item How DSL works
   \item DMT communications
   \item Dynamic Spectrum Management
   \item DSM Levels
   \item Existing/previous DSL simulation systems
   \item Existing/previous DSM implementations
  \end{itemize*}
 \item Solution Research
  \begin{itemize*}   \item Relevant Programming Environments
   \item Software Profiling
   \item Massively Parallel Computing
   \item Parallel Computing Architectures
   \item CUDA software considerations
   \item CUDA hardware considerations
  \end{itemize*}
 \item Development
  \begin{itemize*}   \item Design and implement DSL system model simulator
   \item Implement Optimal Spectrum Balancing (OSB\nomenclature{OSB}{Optimal Spectrum Balancing}) algorithm
   \item Implement Greedy bit-loading algorithm (MIPB\nomenclature{MIPB}{Multi-user Incremental Power Balancing, aka Greedy}) algorithm
   \item Implement Iterative Spectrum Balancing (ISB\nomenclature{ISB}{Iterative Spectrum Balancing}) algorithm
   \item Evaluate system performance through the use of profilers and workload visualisation
   \item Develop single-device GPU versions of above algorithms
   \item Evaluate performance of these algorithms, with respect to CPU\nomenclature{CPU}{Central Processing Unit}-bound versions
   \item If viable, implement multi-GPU versions of above algorithms
  \end{itemize*}
 \item Analysis
   \begin{itemize*}    \item Compare and contrast performance concerns and resource usage of differing implementations
    \item Evaluate system cost concerns, and potential future avenues of development.
   \end{itemize*}
\end{itemize*}%
\lthtmlfigureZ
\lthtmlcheckvsize\clearpage}

\stepcounter{chapter}
\stepcounter{section}
\stepcounter{subsection}
{\newpage\clearpage
\lthtmlinlinemathA{tex2html_wrap_indisplay9194}%
$\displaystyle b=log_2(M)$%
\lthtmlindisplaymathZ
\lthtmlcheckvsize\clearpage}

\stepcounter{subsection}
\stepcounter{subsubsection}
{\newpage\clearpage
\lthtmlinlinemathA{tex2html_wrap_inline9200}%
$ N$%
\lthtmlinlinemathZ
\lthtmlcheckvsize\clearpage}

{\newpage\clearpage
\lthtmlinlinemathA{tex2html_wrap_inline9202}%
$ K$%
\lthtmlinlinemathZ
\lthtmlcheckvsize\clearpage}

{\newpage\clearpage
\lthtmlinlinemathA{tex2html_wrap_inline9207}%
$ V_1$%
\lthtmlinlinemathZ
\lthtmlcheckvsize\clearpage}

{\newpage\clearpage
\lthtmlinlinemathA{tex2html_wrap_inline9209}%
$ V_2$%
\lthtmlinlinemathZ
\lthtmlcheckvsize\clearpage}

{\newpage\clearpage
\lthtmlinlinemathA{tex2html_wrap_inline9211}%
$ d$%
\lthtmlinlinemathZ
\lthtmlcheckvsize\clearpage}

{\newpage\clearpage
\lthtmlinlinemathA{tex2html_wrap_inline9213}%
$ Z$%
\lthtmlinlinemathZ
\lthtmlcheckvsize\clearpage}

{\newpage\clearpage
\lthtmlinlinemathA{tex2html_wrap_inline9215}%
$ R \cdot L$%
\lthtmlinlinemathZ
\lthtmlcheckvsize\clearpage}

{\newpage\clearpage
\lthtmlinlinemathA{tex2html_wrap_inline9217}%
$ Y$%
\lthtmlinlinemathZ
\lthtmlcheckvsize\clearpage}

{\newpage\clearpage
\lthtmlinlinemathA{tex2html_wrap_inline9219}%
$ G \cdot C$%
\lthtmlinlinemathZ
\lthtmlcheckvsize\clearpage}

{\newpage\clearpage
\lthtmlinlinemathA{tex2html_wrap_inline9221}%
$ Z_l$%
\lthtmlinlinemathZ
\lthtmlcheckvsize\clearpage}

{\newpage\clearpage
\lthtmlinlinemathA{tex2html_wrap_inline9223}%
$ Z_s$%
\lthtmlinlinemathZ
\lthtmlcheckvsize\clearpage}

{\newpage\clearpage
\lthtmlinlinemathA{tex2html_wrap_inline9225}%
$ Z_0$%
\lthtmlinlinemathZ
\lthtmlcheckvsize\clearpage}

{\newpage\clearpage
\lthtmlinlinemathA{tex2html_wrap_inline9227}%
$ \sqrt{\frac{Z}{Y}}$%
\lthtmlinlinemathZ
\lthtmlcheckvsize\clearpage}

{\newpage\clearpage
\lthtmlinlinemathA{tex2html_wrap_inline9229}%
$ \gamma$%
\lthtmlinlinemathZ
\lthtmlcheckvsize\clearpage}

{\newpage\clearpage
\lthtmlinlinemathA{tex2html_wrap_inline9231}%
$ \sqrt{Z \cdot Y}$%
\lthtmlinlinemathZ
\lthtmlcheckvsize\clearpage}

{\newpage\clearpage
\lthtmlinlinemathA{tex2html_wrap_indisplay9233}%
$\displaystyle H=\frac{Z_0 \cdot \text{sech}(\gamma d)}{Z_s \cdot [\frac{Z_0}{Z_1} +\text{tanh}(\gamma d)]+Z_0 \cdot[1+\frac{Z_0}{Z_1} \cdot \text{tanh}(\gamma d)]}$%
\lthtmlindisplaymathZ
\lthtmlcheckvsize\clearpage}

{\newpage\clearpage
\lthtmlinlinemathA{tex2html_wrap_indisplay9237}%
$\displaystyle R(f) = \frac{1}{\frac{1}{\sqrt[4]{r^4_{0c}+a_c\cdot f^2}} + \frac{1}{\sqrt[4]{r^4_{0s}+a_s\cdot f^2}}}$%
\lthtmlindisplaymathZ
\lthtmlcheckvsize\clearpage}

{\newpage\clearpage
\lthtmlinlinemathA{tex2html_wrap_inline9239}%
$ r_{0x}$%
\lthtmlinlinemathZ
\lthtmlcheckvsize\clearpage}

{\newpage\clearpage
\lthtmlinlinemathA{tex2html_wrap_inline9241}%
$ a_x$%
\lthtmlinlinemathZ
\lthtmlcheckvsize\clearpage}

{\newpage\clearpage
\lthtmlinlinemathA{tex2html_wrap_indisplay9243}%
$\displaystyle L(f) = \frac{l_0+l_{\infty}(\frac{f}{f_m})^b}{1+(\frac{f}{f_b})^b}$%
\lthtmlindisplaymathZ
\lthtmlcheckvsize\clearpage}

{\newpage\clearpage
\lthtmlinlinemathA{tex2html_wrap_inline9245}%
$ l_x$%
\lthtmlinlinemathZ
\lthtmlcheckvsize\clearpage}

{\newpage\clearpage
\lthtmlinlinemathA{tex2html_wrap_inline9247}%
$ \infty$%
\lthtmlinlinemathZ
\lthtmlcheckvsize\clearpage}

{\newpage\clearpage
\lthtmlinlinemathA{tex2html_wrap_inline9249}%
$ b$%
\lthtmlinlinemathZ
\lthtmlcheckvsize\clearpage}

{\newpage\clearpage
\lthtmlinlinemathA{tex2html_wrap_inline9251}%
$ f_m$%
\lthtmlinlinemathZ
\lthtmlcheckvsize\clearpage}

{\newpage\clearpage
\lthtmlinlinemathA{tex2html_wrap_indisplay9253}%
$\displaystyle C(f)= c_{\infty} + c_0 \cdot f^{-c_e}$%
\lthtmlindisplaymathZ
\lthtmlcheckvsize\clearpage}

{\newpage\clearpage
\lthtmlinlinemathA{tex2html_wrap_inline9255}%
$ C_x$%
\lthtmlinlinemathZ
\lthtmlcheckvsize\clearpage}

{\newpage\clearpage
\lthtmlinlinemathA{tex2html_wrap_inline9259}%
$ 0,e$%
\lthtmlinlinemathZ
\lthtmlcheckvsize\clearpage}

{\newpage\clearpage
\lthtmlinlinemathA{tex2html_wrap_indisplay9261}%
$\displaystyle G(f)=g_0\cdot f^{+g_e}$%
\lthtmlindisplaymathZ
\lthtmlcheckvsize\clearpage}

{\newpage\clearpage
\lthtmlinlinemathA{tex2html_wrap_inline9263}%
$ G_x$%
\lthtmlinlinemathZ
\lthtmlcheckvsize\clearpage}

{\newpage\clearpage
\lthtmlfigureA{figure415}%
\begin{figure}  \centering
  \begin{tabular}{|c|c|c|c|}\hline
    \(r_{0c}=174.56 \Omega\)/km&\(r_{0s}=\infty \Omega\)/km&\(a_c=0.0531\)&\(a_s=0.0\)\\\hline
    \(l_{0}=617.25 \mu H\)/km&\(l_{\infty} = 478.97 \mu H\)/km&\(b=1.1529\)&\(f_m=553.76\)kHz\\\hline
    \(c_{\infty} = 50 \mu F\)/km&\(c_0=0.0\mu F\)/km&\(c_e=0.0\)&\\\hline
    \(g_0 = 234.874 fS\)/km&\(g_e=1.38\)&&\\\hline
  \end{tabular}
  
\end{figure}%
\lthtmlfigureZ
\lthtmlcheckvsize\clearpage}

\stepcounter{subsubsection}
{\newpage\clearpage
\lthtmlinlinemathA{tex2html_wrap_indisplay9274}%
$\displaystyle |H_{FEXT}(f)|^2=N^{0.06}K_{FEXT}f^2 |H_{channel}(f,L)|^2$%
\lthtmlindisplaymathZ
\lthtmlcheckvsize\clearpage}

{\newpage\clearpage
\lthtmlinlinemathA{tex2html_wrap_indisplay9278}%
$\displaystyle |H^{j,k}_{FEXT}(f)|^2=|H^{j,k}_{FEXT}(f)|^2 \times 10^{\frac{X_{dB}}{10}}$%
\lthtmlindisplaymathZ
\lthtmlcheckvsize\clearpage}

{\newpage\clearpage
\lthtmlinlinemathA{tex2html_wrap_inline9280}%
$ X$%
\lthtmlinlinemathZ
\lthtmlcheckvsize\clearpage}

{\newpage\clearpage
\lthtmlinlinemathA{tex2html_wrap_indisplay9282}%
$\displaystyle C(k)=log_2(1+SNR(k))$%
\lthtmlindisplaymathZ
\lthtmlcheckvsize\clearpage}

{\newpage\clearpage
\lthtmlinlinemathA{tex2html_wrap_inline9284}%
$ C(k)$%
\lthtmlinlinemathZ
\lthtmlcheckvsize\clearpage}

{\newpage\clearpage
\lthtmlinlinemathA{tex2html_wrap_inline9288}%
$ N_e$%
\lthtmlinlinemathZ
\lthtmlcheckvsize\clearpage}

{\newpage\clearpage
\lthtmlinlinemathA{tex2html_wrap_indisplay9290}%
$\displaystyle P_e\approx N_eQ\left(\sqrt{\frac{3}{M-1}SNR}\right)$%
\lthtmlindisplaymathZ
\lthtmlcheckvsize\clearpage}

{\newpage\clearpage
\lthtmlinlinemathA{tex2html_wrap_inline9292}%
$ Q(x)$%
\lthtmlinlinemathZ
\lthtmlcheckvsize\clearpage}

{\newpage\clearpage
\lthtmlinlinemathA{tex2html_wrap_inline9294}%
$ x$%
\lthtmlinlinemathZ
\lthtmlcheckvsize\clearpage}

{\newpage\clearpage
\lthtmlinlinemathA{tex2html_wrap_inline9296}%
$ P_e$%
\lthtmlinlinemathZ
\lthtmlcheckvsize\clearpage}

{\newpage\clearpage
\lthtmlinlinemathA{tex2html_wrap_inline9298}%
$ 1e^{-7}$%
\lthtmlinlinemathZ
\lthtmlcheckvsize\clearpage}

{\newpage\clearpage
\lthtmlinlinemathA{tex2html_wrap_indisplay9300}%
$\displaystyle Q(x)=\int_x^{\infty}\frac{1}{\sqrt{2\pi}}e^{-u^2/2} du$%
\lthtmlindisplaymathZ
\lthtmlcheckvsize\clearpage}

{\newpage\clearpage
\lthtmlinlinemathA{tex2html_wrap_inline9302}%
$ erf(x)$%
\lthtmlinlinemathZ
\lthtmlcheckvsize\clearpage}

{\newpage\clearpage
\lthtmlinlinemathA{tex2html_wrap_indisplay9304}%
$\displaystyle Q(x)=\frac{1}{2}\left(1-erf(\frac{x}{\sqrt{2}})\right)$%
\lthtmlindisplaymathZ
\lthtmlcheckvsize\clearpage}

{\newpage\clearpage
\lthtmlinlinemathA{tex2html_wrap_indisplay9306}%
$\displaystyle Q^{-1}(x)=\sqrt{2} erf^{-1}(1-2x)$%
\lthtmlindisplaymathZ
\lthtmlcheckvsize\clearpage}

{\newpage\clearpage
\lthtmlinlinemathA{tex2html_wrap_inline9308}%
$ M$%
\lthtmlinlinemathZ
\lthtmlcheckvsize\clearpage}

{\newpage\clearpage
\lthtmlinlinemathA{tex2html_wrap_indisplay9310}%
$\displaystyle M=1+\frac{SNR}{\frac{1}{3}(Q^{-1}(\frac{P_e}{N_e})^2)}$%
\lthtmlindisplaymathZ
\lthtmlcheckvsize\clearpage}

{\newpage\clearpage
\lthtmlinlinemathA{tex2html_wrap_indisplay9312}%
$\displaystyle b=log_2\left(1+\frac{SNR}{\frac{1}{3}(Q^{-1}(\frac{P_e}{N_e})^2)}\right)$%
\lthtmlindisplaymathZ
\lthtmlcheckvsize\clearpage}

{\newpage\clearpage
\lthtmlinlinemathA{tex2html_wrap_indisplay9314}%
$\displaystyle \Gamma_{uncoded}=\frac{1}{3}\left(Q^{-1}(\frac{P_e}{N_e}\right)^2$%
\lthtmlindisplaymathZ
\lthtmlcheckvsize\clearpage}

{\newpage\clearpage
\lthtmlinlinemathA{tex2html_wrap_indisplay9316}%
$\displaystyle b=log_2\left(1+\frac{SNR}{\Gamma_{uncoded}}\right)$%
\lthtmlindisplaymathZ
\lthtmlcheckvsize\clearpage}

{\newpage\clearpage
\lthtmlinlinemathA{tex2html_wrap_inline9318}%
$ \Gamma$%
\lthtmlinlinemathZ
\lthtmlcheckvsize\clearpage}

{\newpage\clearpage
\lthtmlinlinemathA{tex2html_wrap_inline9320}%
$ \gamma_m$%
\lthtmlinlinemathZ
\lthtmlcheckvsize\clearpage}

{\newpage\clearpage
\lthtmlinlinemathA{tex2html_wrap_inline9322}%
$ \gamma_{eg}$%
\lthtmlinlinemathZ
\lthtmlcheckvsize\clearpage}

{\newpage\clearpage
\lthtmlinlinemathA{tex2html_wrap_indisplay9326}%
$\displaystyle \Gamma=\Gamma_{uncoded}+\gamma_m+\gamma_{eg}$%
\lthtmlindisplaymathZ
\lthtmlcheckvsize\clearpage}

{\newpage\clearpage
\lthtmlinlinemathA{tex2html_wrap_inline9328}%
$ n$%
\lthtmlinlinemathZ
\lthtmlcheckvsize\clearpage}

{\newpage\clearpage
\lthtmlinlinemathA{tex2html_wrap_inline9332}%
$ k$%
\lthtmlinlinemathZ
\lthtmlcheckvsize\clearpage}

{\newpage\clearpage
\lthtmlinlinemathA{tex2html_wrap_indisplay9336}%
$\displaystyle b_n(k)=log_2\left(1+\frac{p_n(k) |h_{nn}(k)|^2}{\Gamma \left( \sigma_n^2(k)+\sum_{j\neq n}^N p_j(k)|h_{jn}(k)|^2 \right)} \right)$%
\lthtmlindisplaymathZ
\lthtmlcheckvsize\clearpage}

{\newpage\clearpage
\lthtmlinlinemathA{tex2html_wrap_inline9338}%
$ h_{ij}(k)$%
\lthtmlinlinemathZ
\lthtmlcheckvsize\clearpage}

{\newpage\clearpage
\lthtmlinlinemathA{tex2html_wrap_inline9342}%
$ i$%
\lthtmlinlinemathZ
\lthtmlcheckvsize\clearpage}

{\newpage\clearpage
\lthtmlinlinemathA{tex2html_wrap_inline9344}%
$ j$%
\lthtmlinlinemathZ
\lthtmlcheckvsize\clearpage}

{\newpage\clearpage
\lthtmlinlinemathA{tex2html_wrap_inline9354}%
$ p_n(k)$%
\lthtmlinlinemathZ
\lthtmlcheckvsize\clearpage}

{\newpage\clearpage
\lthtmlinlinemathA{tex2html_wrap_inline9366}%
$ \sigma^2_n(k)$%
\lthtmlinlinemathZ
\lthtmlcheckvsize\clearpage}

{\newpage\clearpage
\lthtmlinlinemathA{tex2html_wrap_inline9382}%
$ f(b_n(k))=\Gamma(2^{b_n(k)}-1)$%
\lthtmlinlinemathZ
\lthtmlcheckvsize\clearpage}

{\newpage\clearpage
\lthtmlinlinemathA{tex2html_wrap_indisplay9384}%
$\displaystyle p_n(k)-f(b_n(k))\sum_{j\neq n}^N p_j(k) \frac{|h_{jn}(k)|^2}{|h_{nn}(k)|^2}=f(b_n(k))\frac{\sigma_n^2(k)}{|h_{nn}(k)|^2}$%
\lthtmlindisplaymathZ
\lthtmlcheckvsize\clearpage}

{\newpage\clearpage
\lthtmlinlinemathA{tex2html_wrap_indisplay9386}%
$\displaystyle A(k)P(k)=B(k)$%
\lthtmlindisplaymathZ
\lthtmlcheckvsize\clearpage}

{\newpage\clearpage
\lthtmlinlinemathA{tex2html_wrap_indisplay9388}%
$\displaystyle A(k)_{ij}=\begin{cases} 1, & \mbox{for } i = j \\\frac{-f(b_i(k))|h_{ji}|^2}{|h_{ii}|^2}, & \mbox{for } i \neq j \end{cases}$%
\lthtmlindisplaymathZ
\lthtmlcheckvsize\clearpage}

{\newpage\clearpage
\lthtmlinlinemathA{tex2html_wrap_indisplay9390}%
$\displaystyle P(k)=[p_1(k) \dots p_i(k) \dots p_N(k)]^T$%
\lthtmlindisplaymathZ
\lthtmlcheckvsize\clearpage}

{\newpage\clearpage
\lthtmlinlinemathA{tex2html_wrap_indisplay9392}%
$\displaystyle B(k)=\left[\frac{f(b_1(k)) \sigma^2_1}{|h_{11}|^2} \dots \frac{f(b_i(k)) \sigma^2_i}{|h_{ii}|^2} \dots \frac{f(b_N(k))\sigma^2_N}{|h_{NN}|^2}\right]^T$%
\lthtmlindisplaymathZ
\lthtmlcheckvsize\clearpage}

{\newpage\clearpage
\lthtmlinlinemathA{tex2html_wrap_inline9394}%
$ B(k)$%
\lthtmlinlinemathZ
\lthtmlcheckvsize\clearpage}

{\newpage\clearpage
\lthtmlinlinemathA{tex2html_wrap_inline9396}%
$ b(k$%
\lthtmlinlinemathZ
\lthtmlcheckvsize\clearpage}

{\newpage\clearpage
\lthtmlinlinemathA{tex2html_wrap_inline9400}%
$ P(k)$%
\lthtmlinlinemathZ
\lthtmlcheckvsize\clearpage}

{\newpage\clearpage
\lthtmlinlinemathA{tex2html_wrap_inline9410}%
$ R_1$%
\lthtmlinlinemathZ
\lthtmlcheckvsize\clearpage}

{\newpage\clearpage
\lthtmlinlinemathA{tex2html_wrap_inline9412}%
$ R_2)$%
\lthtmlinlinemathZ
\lthtmlcheckvsize\clearpage}

{\newpage\clearpage
\lthtmlinlinemathA{tex2html_wrap_indisplay9417}%
$\displaystyle \max\limits_{\{p_n\in P_n\}_n} \sum_n w_n R_n$%
\lthtmlindisplaymathZ
\lthtmlcheckvsize\clearpage}

{\newpage\clearpage
\lthtmldisplayA{displaymath9423}%
\begin{displaymath}\begin{array}{rc} \sum\limits_{n=1}^N w_n \sum\limits_{k=1}^K & log_2 \left(1+\frac{p_n(k_|h_{nn}(k)|^2}{\Gamma\left(\sigma_n^2(k)+\sum\limits_{m\neq n}^N p_m(k)|h_{jm}(k)|^2\right)}\right)\\&s.t.\sum_{k=1}^Kp_n(k) \leq P_{max,n} \forall n \end{array}\end{displaymath}%
\lthtmldisplayZ
\lthtmlcheckvsize\clearpage}

\stepcounter{subsection}
\stepcounter{subsection}
{\newpage\clearpage
\lthtmldisplayA{displaymath9463}%
\begin{displaymath}\begin{array}{rcl}   max R&=&\sum_{k=1}^Kb(k)\\s.t.&&\sum\limits_{k=1}^Kp(k)\leq P_{budget}   \end{array}\end{displaymath}%
\lthtmldisplayZ
\lthtmlcheckvsize\clearpage}

{\newpage\clearpage
\lthtmldisplayA{displaymath9465}%
\begin{displaymath}\begin{array}{rcl}   min &&\sum\limits_{k=1}^Kp(k)\\s.t.&&\sum\limits_{k=1}^Kb(k)\geq B_{budget}   \end{array}\end{displaymath}%
\lthtmldisplayZ
\lthtmlcheckvsize\clearpage}

{\newpage\clearpage
\lthtmlinlinemathA{tex2html_wrap_inline9469}%
$ \Delta p(k)$%
\lthtmlinlinemathZ
\lthtmlcheckvsize\clearpage}

\stepcounter{subsubsection}
{\newpage\clearpage
\lthtmlfigureA{figure727}%
\begin{figure}\begin{algorithmic}
\REPEAT
\FOR{n=1...N}
\STATE Execute LC Algorithm with power budget \(P_n\) on line \(n\)
\IF{\(R_n>R_n^{target}\)}
\STATE \(P_n=P_n-\gamma\)
\ELSE
\STATE \(P_n=P_n+\gamma\)
\ENDIF
\ENDFOR
\UNTIL{convergence}
\end{algorithmic}

\end{figure}%
\lthtmlfigureZ
\lthtmlcheckvsize\clearpage}

\stepcounter{subsection}
\stepcounter{subsubsection}
{\newpage\clearpage
\lthtmlinlinemathA{tex2html_wrap_inline9483}%
$ L$%
\lthtmlinlinemathZ
\lthtmlcheckvsize\clearpage}

{\newpage\clearpage
\lthtmlinlinemathA{tex2html_wrap_indisplay9487}%
$\displaystyle L=\sum\limits_{n=1}^N w_n \sum\limits_{k=1}^K log_2 \left(1+\frac{p_n(k_|h_{nn}(k)|^2}{\Gamma\left(\sigma_n^2(k)+\sum\limits_{m\neq n}^N p_m(k)|h_{jm}(k)|^2\right)}\right)-\lambda_n\left(\sum\limits_{k=1}^K p_n(k)\right)$%
\lthtmlindisplaymathZ
\lthtmlcheckvsize\clearpage}

{\newpage\clearpage
\lthtmlinlinemathA{tex2html_wrap_inline9489}%
$ \lambda$%
\lthtmlinlinemathZ
\lthtmlcheckvsize\clearpage}

{\newpage\clearpage
\lthtmlinlinemathA{tex2html_wrap_indisplay9495}%
$\displaystyle L(k)\nomenclature{\(L_k\)}{The Lagrangian sum of a bitloaded channel}=\sum\limits_{n=1}^Nw_nb_n(k)-\sum\limits_{n=1}^N\lambda_np_n(k)$%
\lthtmlindisplaymathZ
\lthtmlcheckvsize\clearpage}

{\newpage\clearpage
\lthtmlinlinemathA{tex2html_wrap_inline9497}%
$ b_n$%
\lthtmlinlinemathZ
\lthtmlcheckvsize\clearpage}

{\newpage\clearpage
\lthtmlinlinemathA{tex2html_wrap_inline9505}%
$ b_{max}$%
\lthtmlinlinemathZ
\lthtmlcheckvsize\clearpage}

{\newpage\clearpage
\lthtmlinlinemathA{tex2html_wrap_inline9509}%
$ b_{max}^{N}-1$%
\lthtmlinlinemathZ
\lthtmlcheckvsize\clearpage}

{\newpage\clearpage
\lthtmlfigureA{figure776}%
\begin{figure}\begin{algorithmic}
\REPEAT
\REPEAT
\FOR{\(k=1\dots K\)}
\STATE \(\arg\max_{b_(k)}L(k)=\sum_{n=1}^Nw_nb_n(k)-\sum_{n=1}^N\lambda_np_n(k)\)
\STATE Solve by N-d exhaustive search
\ENDFOR
\STATE \(\lambda_n=\lambda_n+\epsilon(\sum_{k=1}^Kp_n(k)-P_n^{max})\)
\UNTIL{\(\lambda\) convergence}
\STATE \(w_n=w_n+\epsilon(\sum_{k=1}^Kb_n(k)-R_n^{target})\)
\UNTIL{\(w\) convergence}
\end{algorithmic}

\end{figure}%
\lthtmlfigureZ
\lthtmlcheckvsize\clearpage}

\stepcounter{subsubsection}
{\newpage\clearpage
\lthtmlinlinemathA{tex2html_wrap_inline9521}%
$ O(b_{\text{max}}^N)$%
\lthtmlinlinemathZ
\lthtmlcheckvsize\clearpage}

{\newpage\clearpage
\lthtmlinlinemathA{tex2html_wrap_inline9523}%
$ O(N^2)$%
\lthtmlinlinemathZ
\lthtmlcheckvsize\clearpage}

{\newpage\clearpage
\lthtmlfigureA{figure800}%
\begin{figure}\begin{algorithmic}
\REPEAT
\FOR{\(k=1\dots K\)}
\REPEAT
\FOR{\(n=1\dots N\)}
\STATE Fix \(b_m(k)\forall m \neq n\)
\STATE \(\arg\max_{b_(k)}L(k)=\sum_{n=1}^Nw_nb_n(k)-\sum_{n=1}^N\lambda_np_n(k)\)
\STATE \(\lambda_n=\lambda_n+\epsilon(\sum_{k=1}^Kp_n(k)-P_n^{max})\)
\STATE Solve by 1-d exhaustive search
\ENDFOR
\UNTIL{\(\lambda\) convergence}
\ENDFOR
\STATE \(w_n=w_n+\epsilon(\sum_{k=1}^Kb_n(k)-R_n^{target})\)
\UNTIL{\(w\) convergence}
\end{algorithmic}

\end{figure}%
\lthtmlfigureZ
\lthtmlcheckvsize\clearpage}

\stepcounter{subsubsection}
{\newpage\clearpage
\lthtmlinlinemathA{tex2html_wrap_inline9529}%
$ \Delta p_m(k)$%
\lthtmlinlinemathZ
\lthtmlcheckvsize\clearpage}

{\newpage\clearpage
\lthtmlinlinemathA{tex2html_wrap_inline9531}%
$ m$%
\lthtmlinlinemathZ
\lthtmlcheckvsize\clearpage}

{\newpage\clearpage
\lthtmlinlinemathA{tex2html_wrap_inline9535}%
$ \Delta p_n(k)$%
\lthtmlinlinemathZ
\lthtmlcheckvsize\clearpage}

{\newpage\clearpage
\lthtmlinlinemathA{tex2html_wrap_indisplay9537}%
$\displaystyle C(m,k)=\sum\limits_{n=1}^N\left(\underset{p_n(k)}{b(k)+1} - \underset{p_n(k)}{b(k)}\right)$%
\lthtmlindisplaymathZ
\lthtmlcheckvsize\clearpage}

\stepcounter{subsubsection}
{\newpage\clearpage
\lthtmlfigureA{figure832}%
\begin{figure}\begin{algorithmic}
\REPEAT
\STATE \(\text{argmin}_{n,k} C\)
\STATE \(b_n{k}=b_n{k}+1\)
\FOR{\(n=1\dots N\)}
  \STATE \(\delta p_{n,k}=\left(\underset{p_n(k)}{b(k)+1} - \underset{p_n(k)}{b(k)}\right)\)
\ENDFOR
\FOR{\(n=1\dots N\)}
  \FOR{\(k=1\dots K\)}
    \STATE \(C_{n,k}=\sum\limits_{n=1}^N \text{wp}(n)\times\delta p_{n,k}\)
    \STATE Where wp\(n\) is a power penalty function
  \ENDFOR
\ENDFOR
\UNTIL All tones full
\end{algorithmic}

\end{figure}%
\lthtmlfigureZ
\lthtmlcheckvsize\clearpage}

{\newpage\clearpage
\lthtmlfigureA{figure854}%
\begin{figure}  \begin{tabularx}{1.1\textwidth}{|X|c|c|c|c|c|c|c|c|}\hline
  &&\multicolumn{7}{|c|}{N-User Runtimes (s/sec,m/min,h/hrs)}\\\hline
  Algorithm&Complexity&2&3&4&5&6&7&8\\\hline
  OSB&\(O(K b_{\text{max}}^N N^3)\)&35.65s&3.24h&*&*&*&*&*\\\hline
  OSB (Cached)&\(K b_{\text{max}}^N N^3)\)&14.6s&2.12h&*&*&*&*&*\\\hline
  ISB&\(O(K N^2 N^3)\)&32.31s&16.22m&3.93hrs&12.06h&8.69h&28.46h&*\\\hline
  ISB (Cached)&\(O(K N^2 N^3)\)&2.73s&53.57s&9.35m&30.04m&26.74m&1.42h&*\\\hline
  BBOSB&\(O(K b_{\text{max}}^N N^3)\)&7.12s&15.73m&13.04h&*&*&*&*\\\hline
  BBOSB (Cached)&\(O(K b_{\text{max}}^N N^3)\)&2.88s&4.2m&3.12h&*&*&*&*\\\hline
  MIPB Bisection&\(O(B_{\text{total}} N^2 N^3)\)&0.26s&9.99s&35.6s&3.71m&12.88m&6.36m&51.72m\\\hline
  MIPB (Cached)&\(O(B_{\text{total}} N^2 N^3)\)&0.19s&3.89s&10.96s&49.97s&*&*&*\\\hline
\end{tabularx}

\end{figure}%
\lthtmlfigureZ
\lthtmlcheckvsize\clearpage}

\stepcounter{subsection}
\stepcounter{section}
\stepcounter{subsection}
\stepcounter{subsubsection}
\stepcounter{subsubsection}
\stepcounter{subsection}
\stepcounter{subsubsection}
{\newpage\clearpage
\lthtmlinlinemathA{tex2html_wrap_inline9615}%
$ \alpha$%
\lthtmlinlinemathZ
\lthtmlcheckvsize\clearpage}

{\newpage\clearpage
\lthtmlinlinemathA{tex2html_wrap_indisplay9619}%
$\displaystyle S=\frac{1}{\alpha}$%
\lthtmlindisplaymathZ
\lthtmlcheckvsize\clearpage}

{\newpage\clearpage
\lthtmlinlinemathA{tex2html_wrap_inline9624}%
$ P$%
\lthtmlinlinemathZ
\lthtmlcheckvsize\clearpage}

{\newpage\clearpage
\lthtmlinlinemathA{tex2html_wrap_indisplay9626}%
$\displaystyle S(P)=P-\alpha(P-1)$%
\lthtmlindisplaymathZ
\lthtmlcheckvsize\clearpage}

\stepcounter{subsection}
\stepcounter{subsection}
{\newpage\clearpage
\lthtmlfigureA{figure1013}%
\begin{figure}  \centering
    \begin{lstlisting}[numbers=left, language=C, numberstyle=\tiny , numbersep=8pt]
    //Setup dimensions of grids and blocks
    dim3 blocksPerGrid(65535,1,1);
    dim3 threadsPerBlock(64,8,1);
\par
//Invoke Kernel
    kernelfunction<<<blocksPerGrid,threadsPerBlock>>>(*functionarguments);
    \end{lstlisting}
  
\end{figure}%
\lthtmlfigureZ
\lthtmlcheckvsize\clearpage}

{\newpage\clearpage
\lthtmlinlinemathA{tex2html_wrap_inline9652}%
$ 2^{16}\times 2^{10} = 2^{27}$%
\lthtmlinlinemathZ
\lthtmlcheckvsize\clearpage}

{\newpage\clearpage
\lthtmlfigureA{figure1026}%
\begin{figure}  \centering
    \begin{lstlisting}[numbers=left, language=C, numberstyle=\tiny , numbersep=8pt]
    __global__ void multArray(float *array, float multiplier){
      int 1Dindex = blockIdx.x*blockDim.x+threadIdx.x;
      array[1Dindex]=array[1Dindex]*multiplier;
    }
    \end{lstlisting}
  
\end{figure}%
\lthtmlfigureZ
\lthtmlcheckvsize\clearpage}

{\newpage\clearpage
\lthtmlfigureA{table1033}%
\begin{table}\begin{tabularx}{\textwidth}{|X|c|c|}
Function Declaration&Executed by&Callable From\\\hline
\_\_device\_\_ void someKernel&device&device\\
\_\_global\_\_ void someKernel&device&host\\
\_\_host\_\_ void someKernel&host&host\\
\end{tabularx}
\end{table}%
\lthtmlfigureZ
\lthtmlcheckvsize\clearpage}

{\newpage\clearpage
\lthtmlinlinemathA{tex2html_wrap_inline9662}%
$ 2^{16}*2^{16}*2^{10}*2^{10}=2^{52}$%
\lthtmlinlinemathZ
\lthtmlcheckvsize\clearpage}

{\newpage\clearpage
\lthtmlinlinemathA{tex2html_wrap_inline9664}%
$ 2^{27}$%
\lthtmlinlinemathZ
\lthtmlcheckvsize\clearpage}

{\newpage\clearpage
\lthtmlfigureA{figure1049}%
\begin{figure}  \centering
    \begin{lstlisting}[numbers=left, language=C, numberstyle=\tiny , numbersep=8pt]
    __global__ void multMatrix(float *matrix, float multiplier){
      int x_index = blockIdx.x*blockDim.x+threadIdx.x*;
      int y_index = blockIdx.y*blockDim.y+threadIdx.y*;
      matrix[x_index][y_index]=matrix[x_index][y_index]*multiplier;
    }
    \end{lstlisting}
  
\end{figure}%
\lthtmlfigureZ
\lthtmlcheckvsize\clearpage}

\stepcounter{subsection}
{\newpage\clearpage
\lthtmlfigureA{table1073}%
\begin{table}\begin{tabularx}{\textwidth}{|c|c|c|c|X|c|}\hline
Memory&Scope&Lifetime&R/W&Usage&Speed\\\hline
Register&Thread&Kernel&R/W&Automatic variables other than arrays&Very Fast\\
Local&Thread&Kernel&R/W&Automatic Array Variables&Very Fast\\\hline
Shared&Block&Kernel&R/W&\_\_shared\_\_&Fast\\\hline
Global&Grid&Application&R/W&Default&Very Slow\\
Constant&Grid&Application&R&\_\_constant\_\_&Slow\\\hline
\end{tabularx}

\end{table}%
\lthtmlfigureZ
\lthtmlcheckvsize\clearpage}

{\newpage\clearpage
\lthtmlinlinemathA{tex2html_wrap_inline9690}%
$ A, B, C$%
\lthtmlinlinemathZ
\lthtmlcheckvsize\clearpage}

{\newpage\clearpage
\lthtmlfigureA{figure1094}%
\begin{figure}  \centering
    \begin{lstlisting}[numbers=left, language=C, numberstyle=\tiny , numbersep=8pt]
      __global__ void matmulNaive(float *A, float *B, float *C, int WIDTH){
        Tx = threadIdx.x; Ty = threadIdx.y;
        Bx = blockIdx.x; By = blockIdx.y;
\par
X = Bx * blockDim.x + Tx;
        Y = By * blockDim.y + Ty;
\par
idxA = Y * WIDTH;   
        idxB = X;
        idxC = Y * WIDTH + X;
\par
Csub = 0.0;
\par
for (i=0; I < WIDTH; i++) {
          Csub += A[idxA] * B[idxB];
          idxA += 1;
          idxB += WIDTH;
        }
\par
C[idxC] = Csub;
      }
    \end{lstlisting}
  
\end{figure}%
\lthtmlfigureZ
\lthtmlcheckvsize\clearpage}

{\newpage\clearpage
\lthtmlfigureA{figure1108}%
\begin{figure}  \centering
    \begin{lstlisting}[numbers=left, language=C, numberstyle=\tiny , numbersep=8pt]
\par
__shared__ float As[blockDim.x][blockDim.y];
        __shared__ float Bs[blockDim.y][blockDim.x];
        Csub = 0.0;
        Tx = threadIdx.x; Ty = threadIdx.y;
        Bx = blockIdx.x; By = blockIdx.y;
\par
X = Bx * blockDim.x + Tx;
        Y = By * blockDim.y + Ty;
\par
idxA = Y * WIDTH;   
        idxB = X;
        idxC = Y * WIDTH + X;
\par
while (idxA<WIDTH) {  // iterate across tiles
          As[Ty][Tx] = A[idxA];
          Bs[Ty][Tx] = B[idxB];
          idxA += blockDim.x;  idxB += blockDim.y * WIDTH;
          __syncthreads();
          for (i=0; i < 16; i++) {
            Csub += As[Ty][i] * Bs[i][Tx];
            __syncthreads();
          }
        }
        C[idxC] = Csub
    \end{lstlisting}
  
\end{figure}%
\lthtmlfigureZ
\lthtmlcheckvsize\clearpage}

{\newpage\clearpage
\lthtmlinlinemathA{tex2html_wrap_inline9701}%
$ k^{th}$%
\lthtmlinlinemathZ
\lthtmlcheckvsize\clearpage}

{\newpage\clearpage
\lthtmlinlinemathA{tex2html_wrap_inline9705}%
$ N~180$%
\lthtmlinlinemathZ
\lthtmlcheckvsize\clearpage}

\stepcounter{section}
\stepcounter{subsection}
\stepcounter{subsection}
{\newpage\clearpage
\lthtmlinlinemathA{tex2html_wrap_inline9725}%
$ O(2^N)$%
\lthtmlinlinemathZ
\lthtmlcheckvsize\clearpage}

\stepcounter{subsection}
\stepcounter{chapter}
\stepcounter{section}
{\newpage\clearpage
\lthtmlfigureA{enumeratestar2222}%
\begin{enumerate*}
  \item Framework Development and Verification of CPU-bound algorithms
  \begin{enumerate*}    \item Implementation of Object Oriented DSL bundle simulation framework, with software hooks for algorithm interfacing and standardised result and instrumentation storage formatting.
    \item \emph{Verify Channel Matrix generation against known dataset}
    \item Pure-Python implementation of OSB
    \item \emph{Verify OSB bit-loading against known dataset}
    \item Pure-Python implementations of MIPB and ISB
    \item \emph{Verify results against known dataset}
    \item Refactor algorithm object structure to reduce functional duplication
  \end{enumerate*}
  \item Development and Verification of GPU-bound algorithms
  \begin{enumerate*}    \item Single GPU implementation of OSB
    \item \emph{Verify results against CPU version}
    \item Multi-device development and implementation of OSB
    \item \emph{Verify results against CPU version}
    \item Single GPU implementation of ISB
    \item \emph{Verify results against CPU version}
    \item Refactor algorithm object structure to reduce functional duplication
  \end{enumerate*}
\end{enumerate*}%
\lthtmlfigureZ
\lthtmlcheckvsize\clearpage}

\stepcounter{section}
{\newpage\clearpage
\lthtmldisplayA{displaymath9747}%
\begin{displaymath}\begin{array}{l}   L_h=(V_{lt}-X_{lt})\\L_s=\text{abs}(\max(V_{lt},X_{lt})-\min(V_{nt},X_{nt}))\\L_t=(V_{nt}-X_{nt})\\\end{array}\end{displaymath}%
\lthtmldisplayZ
\lthtmlcheckvsize\clearpage}

{\newpage\clearpage
\lthtmlinlinemathA{tex2html_wrap_indisplay9749}%
$\displaystyle (L,F)=\left\{     \begin{array}{l l}       \text{transfer function}(L,F) & \quad \text{if}\  L>0\\1 & \quad \text{if} L\le 0\\\end{array} \right.$%
\lthtmlindisplaymathZ
\lthtmlcheckvsize\clearpage}

{\newpage\clearpage
\lthtmldisplayA{displaymath9751}%
\begin{displaymath}\begin{array}{l}   H_h=\text{insertion loss}(L_h,f_k)\\H_s=\text{insertion loss}(L_s,f_k)\\H_t=\text{insertion loss}(L_t,f_k)\\H_{\text{total}}=(H_h \times H_s \times H_t)\\\end{array}\end{displaymath}%
\lthtmldisplayZ
\lthtmlcheckvsize\clearpage}

{\newpage\clearpage
\lthtmlinlinemathA{tex2html_wrap_inline9756}%
$ N\times N$%
\lthtmlinlinemathZ
\lthtmlcheckvsize\clearpage}

\stepcounter{section}
{\newpage\clearpage
\lthtmlfigureA{figure2321}%
\begin{figure}  \begin{algorithmic}
    \FORALL{channels}
      \REPEAT
        \FORALL{lines}
        \STATE{\(\text{argmax}_{b_k}L(k)\)}
        \STATE{By 1-d exhaustive search}
        \ENDFOR
      \UNTIL{Bit-load Convergence}
    \ENDFOR
  \end{algorithmic}
  
\end{figure}%
\lthtmlfigureZ
\lthtmlcheckvsize\clearpage}

{\newpage\clearpage
\lthtmlfigureA{figure2333}%
\begin{figure}  \begin{algorithmic}
    \REPEAT
      \FORALL{channels}
        \FORALL{lines}
        \STATE{\(\text{argmax}_{b_k}L(k)\)}
        \STATE{By 1-d exhaustive search}
        \ENDFOR
      \ENDFOR
    \UNTIL{Bit-load Convergence}
  \end{algorithmic}
  
\end{figure}%
\lthtmlfigureZ
\lthtmlcheckvsize\clearpage}

\stepcounter{section}
\stepcounter{subsection}
\stepcounter{subsubsection}
{\newpage\clearpage
\lthtmlinlinemathA{tex2html_wrap_inline9793}%
$ \Delta p$%
\lthtmlinlinemathZ
\lthtmlcheckvsize\clearpage}

{\newpage\clearpage
\lthtmlinlinemathA{tex2html_wrap_inline9797}%
$ 4$%
\lthtmlinlinemathZ
\lthtmlcheckvsize\clearpage}

{\newpage\clearpage
\lthtmlinlinemathA{tex2html_wrap_inline9805}%
$ 2^{16}$%
\lthtmlinlinemathZ
\lthtmlcheckvsize\clearpage}

\stepcounter{subsection}
\stepcounter{subsection}
\stepcounter{subsubsection}
{\newpage\clearpage
\lthtmlinlinemathA{tex2html_wrap_inline9822}%
$ =B_{\text{max}}\times K$%
\lthtmlinlinemathZ
\lthtmlcheckvsize\clearpage}

\stepcounter{subsubsection}
{\newpage\clearpage
\lthtmlfigureA{figure2407}%
\begin{figure}  \lstinputlisting[language=Python]{sourcefiles/workload-calc.py}
  
\end{figure}%
\lthtmlfigureZ
\lthtmlcheckvsize\clearpage}

\stepcounter{subsubsection}
\stepcounter{section}
\stepcounter{subsection}
{\newpage\clearpage
\lthtmlfigureA{itemizestar2420}%
\begin{itemize*}
\item{'lk\_osb\_prepare\_permutations': Generate A and B matrices for PSD calculation of all bit permutations}
\item{'osb\_solve': Solve all permutation's PSD's in parallel}
\item{'lk\_max\_permutations': Solve all permutation's PSD's in parallel}
\end{itemize*}%
\lthtmlfigureZ
\lthtmlcheckvsize\clearpage}

\stepcounter{subsubsection}
\stepcounter{subsection}
{\newpage\clearpage
\lthtmlinlinemathA{tex2html_wrap_inline9853}%
$ [K|B_{\text{max}}]$%
\lthtmlinlinemathZ
\lthtmlcheckvsize\clearpage}

\stepcounter{subsubsection}
\stepcounter{subsection}
\stepcounter{chapter}
\stepcounter{section}
\stepcounter{section}
\stepcounter{subsection}
{\newpage\clearpage
\lthtmlfigureA{figure2767}%
\begin{figure}
\centering 
  \begin{tabularx}{0.6\textwidth}{|c|X|c|c|c|c|}
  \hline
  &CPU time&\multicolumn{4}{|c|}{GPU Count runtime (s)}\\
  N&&1&2&3&4\\\hline
  2&73.881&13.030&9.005&9.224&9.100\\
  3&651.977&18.737&13.523&12.851&13.100\\
  4&3118.691&334.820&175.373&122.746&98.096\\
  5&NA&6320.587&3175.932&2136.499&1605.589\\\hline
  \end{tabularx}


\end{figure}%
\lthtmlfigureZ
\lthtmlcheckvsize\clearpage}

\stepcounter{subsection}
{\newpage\clearpage
\lthtmlinlinemathA{tex2html_wrap_inline9873}%
$ S_p=\frac{T_1}{T_p}$%
\lthtmlinlinemathZ
\lthtmlcheckvsize\clearpage}

{\newpage\clearpage
\lthtmlinlinemathA{tex2html_wrap_inline9875}%
$ T_1$%
\lthtmlinlinemathZ
\lthtmlcheckvsize\clearpage}

{\newpage\clearpage
\lthtmlinlinemathA{tex2html_wrap_inline9877}%
$ T_p$%
\lthtmlinlinemathZ
\lthtmlcheckvsize\clearpage}

{\newpage\clearpage
\lthtmlinlinemathA{tex2html_wrap_inline9879}%
$ E_p=\frac{S_p}{P}$%
\lthtmlinlinemathZ
\lthtmlcheckvsize\clearpage}

{\newpage\clearpage
\lthtmlinlinemathA{tex2html_wrap_inline9885}%
$ E_p=\frac{S_p}{240 \times P}$%
\lthtmlinlinemathZ
\lthtmlcheckvsize\clearpage}

\stepcounter{section}
\stepcounter{subsection}
{\newpage\clearpage
\lthtmlfigureA{figure2803}%
\begin{figure}
\centering 
  \begin{tabularx}{0.5\textwidth}{|c|X|c|c|c|c|}
    \hline
  &CPU time&\multicolumn{4}{|c|}{GPU Count runtime (s)}\\
  N&&1&2&3&4\\\hline
  2&27.625&0.363&0.000&0.000&0.000\\
  3&76.384&0.528&0.497&0.575&0.625\\
  4&225.278&0.940&0.863&0.944&1.020\\
  5&650.274&1.644&1.277&1.266&1.337\\\hline
  \end{tabularx}


\end{figure}%
\lthtmlfigureZ
\lthtmlcheckvsize\clearpage}

\stepcounter{subsection}
\stepcounter{chapter}
\stepcounter{section}
\stepcounter{section}
\stepcounter{section}
\stepcounter{section}
\stepcounter{section}

\renewcommand{\bibname}{References}

\renewcommand{\thelstlisting}{\Alph{lstlisting}}

\renewcommand{\figurename}{Appendix}

\renewcommand{\}{}
\appendix

\end{document}
